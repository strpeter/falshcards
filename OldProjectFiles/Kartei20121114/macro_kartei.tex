% Macros for usage in the script

% Dirac notation
	\newcommand{\ket}[1]{\vert  #1 \rangle}
	\newcommand{\bra}[1]{\langle #1 |}
	\newcommand{\inprod}[2]{\langle #1 | #2 \rangle}
	\newcommand{\proj}[2]{\ket{#1}\bra{#2}}
	\newcommand{\pure}[1]{\proj{#1}{#1}}
 
%Set
	\newcommand{\set}[1]{\left\{ #1 \right\}}
% Average
	\newcommand{\avg}[1]{\langle #1 \rangle}

% % Absolute value
	\DeclarePairedDelimiter\abs{\lvert}{\rvert}
	\makeatletter
	\let\oldabs\abs
	\def\abs{\@ifstar{\oldabs}{\oldabs*}}

% % Hamiltonian
	\newcommand{\h}[1]{\hat{H}_{#1}  }

% % Probability
    \newcommand{\sprob}[1]{\text{Prob}\left(#1\right)  }
    
% % Trace
	\newcommand{\ttr}[1]{\mbox{Tr}\left(#1\right)  }
    
% % Endomorphism
	\newcommand{\End}[1]{\mathrm{End}(#1)}

% % Commutator
	\newcommand{\commute}[2]{[#1, #2]}

% % Vectors
	\newcommand{\vect}[1]{\mathbf{#1}}
	\renewcommand{\vec}[1]{\mathbf{#1}}
	\newcommand{\vecop}[1]{\hat{\mathbf{#1}}}

% Hilbert spaces
	\newcommand{\hilbert}{\mathcal{H}}
	\newcommand{\composed}{\mathcal{H}_A  \otimes \mathcal{H}_R}

% Spinors (a bit of cheating)
	\newcommand{\spinor}[2]{ {#1 \choose #2} }
	\newcommand{\one}{\! \! \uparrow}
	\newcommand{\zero}{\! \! \downarrow}
	
%Trace
	\newcommand{\tr}{ \mbox{Tr}  }

% Probability
	%\newcommand{\prob}{\mathcal{P}}
	%\newcommand{\prob}[1]{\mbox{Prob} \left[#1\right]}

%Identity operator
	\newcommand{\id}{\mathbbm{1}}
	%\newcommand{\id}{id}
	\newcommand{\cj}{Choi-Jamio\l{}kowski }

%Set of unitaries
	\newcommand{\UU}{\mathcal{U}}

% Entropies
	\newcommand{\hmin}{ H_{\min} }
	\newcommand{\hmax}{ H_{\max} }  
	\newcommand{\renyi}{R\' enyi }

% Max eg value
	\newcommand{\lmax}{ \lambda_{\max} }

% Operators
	\newcommand{\cnot}{\mbox{ CNOT }}
	\newcommand{\rank}{\mbox{rank }}
	\newcommand{\supp}{\operatorname{supp}}

% Matrices
	\newcommand{\matrixTwo}[1]{
		\left( 
			\begin{array}{c c}
				#1
			\end{array}
		\right)
		}
	%melhora este:
	\newcommand{\matrixN}[2]{
		\left( 
			\begin{array}{#1}
				#2
			\end{array}
		\right)
		}

% Long probabilities
	\newcommand{\prob}[2]{
		\Pr_{#1}
		\left\{
			\begin{array}{l}
				#2
			\end{array}
		\right\}
		}

\newcommand*{\eps}{\varepsilon}
\newcommand*{\half}{\frac{1}{2}}

%Sets and spaces
\newcommand{\R}{\mathbbm{R}}
\newcommand{\N}{\mathbbm{N}}
%TPCPMs
\newcommand*{\I}{\mathcal{I}}
\newcommand*{\X}{\mathcal{X}}
\newcommand*{\E}{\mathcal{E}}
\newcommand*{\cM}{\mathcal{M}}
\newcommand{\cT}{\mathcal{T}}
\newcommand{\OO}[1]{\mathcal{O} \left( #1 \right) }
% % \newcommand{\ham}{\mathcal{H}}


% Partial derivatives
\newcommand{\pd}[1]{ \frac{\partial}{\partial #1} }     %first derivative
\newcommand{\pdd}[1]{ \frac{\partial^2}{\partial {#1}^2 } }   %second derivative

%..................................

% define boxed equations in all variants with package empheq with e.g. color blue:
\definecolor{myblue}{rgb}{.8, .8, 1}

\newlength\mytemplen
\newsavebox\mytempbox
\makeatletter
\newcommand\mybluebox{%
    \@ifnextchar[%]
       {\@mybluebox}%
       {\@mybluebox[0pt]}}

\def\@mybluebox[#1]{%
    \@ifnextchar[%]
       {\@@mybluebox[#1]}%
       {\@@mybluebox[#1][0pt]}}

\def\@@mybluebox[#1][#2]#3{
    \sbox\mytempbox{#3}%
    \mytemplen\ht\mytempbox
    \advance\mytemplen #1\relax
    \ht\mytempbox\mytemplen
    \mytemplen\dp\mytempbox
    \advance\mytemplen #2\relax
    \dp\mytempbox\mytemplen
    \colorbox{myblue}{\hspace{1em}\usebox{\mytempbox}\hspace{1em}}}
\makeatother


%..................................................

% Colours
	\newcommand{\red}[1]{\textcolor{Red}{#1}}
	\newcommand{\blue}[1]{\textcolor{Blue}{#1}}
	\newcommand{\green}[1]{\textcolor{OliveGreen}{#1}}
	\newcommand{\pink}[1]{\textcolor{Magenta}{#1}}
	\newcommand{\yellow}[1]{\textcolor{Yellow}{#1}}
	\newcommand{\orange}[1]{\textcolor{Orange}{#1}}

% Little things
	\newcommand{\ie}{i.e., }
	\newcommand{\etal}{et al.\ }
	\newcommand{\fred}{Fr\' ed\' eric}
	%\newcommand{\iid}{i.i.d.\  }

%side note
\newcommand{\flag}[1]{\green{ [#1]}}


%.................................................

% Referencing equations and figures
	\newcommand{\eref}[1]{Eq.\ (\ref{#1})}
	\newcommand{\fref}[1]{Fig.\ \ref{#1}}


%.................................................

% Tikz settings
\tikzset{
	% Feynman Diagrams
	photon/.style={decorate, decoration={snake}},
	% photondir/.style={decorate, decoration={snake}, segment amplitude=.8mm, segment length=4mm, line after s    nake=1.3mm, ->},
    photondir/.style={decorate, decoration={snake}, -stealth},
	fermion/.style={postaction={decorate},decoration={markings,mark=at position .55 with {\arrow{>}}}},
	gluon/.style={decorate,decoration={coil,amplitude=4pt,segment length=5pt}},
	boson/.style={dashed},
}

