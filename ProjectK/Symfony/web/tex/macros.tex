% Macros for usage in the script

%Dirac notation
\newcommand{\ket}[1]{\vert  #1 \rangle}
\newcommand{\bra}[1]{\langle #1 |}
\newcommand{\inprod}[2]{\langle #1 | #2 \rangle}
\newcommand{\proj}[2]{\ket{#1}\bra{#2}}
\newcommand{\pure}[1]{\proj{#1}{#1}}

%Average
\newcommand{\avg}[1]{\langle #1 \rangle}

%Absolute value
\DeclarePairedDelimiter\abs{\lvert}{\rvert}
\makeatletter
\let\oldabs\abs
\def\abs{\@ifstar{\oldabs}{\oldabs*}}

%Hamiltonian
% \newcommand{\ham}{\mathcal{H}}
\newcommand{\h}[1]{\hat{H}_{#1}}

%Probability
\newcommand{\sprob}[1]{\text{Prob}\left(#1\right)  }
   
%Trace
\newcommand{\ttr}[1]{\mbox{Tr}\left(#1\right)  }
    
%Endomorphism
\newcommand{\End}[1]{\mathrm{End}(#1)}

%Commutator
\newcommand{\commute}[2]{[#1, #2]}

%Vectors
\newcommand{\vect}[1]{\mathbf{#1}}
\renewcommand{\vec}[1]{\mathbf{#1}}
\newcommand{\vecop}[1]{\hat{\mathbf{#1}}}

%Hilbert spaces
\newcommand*{\hil}{\mathcal{H}}
\newcommand*{\composed}{\mathcal{H}_A\otimes\mathcal{H}_R}

%Spinors (a bit of cheating)
\newcommand{\spinor}[2]{ {#1 \choose #2} }
\newcommand{\one}{\! \! \uparrow}
\newcommand{\zero}{\! \! \downarrow}
	
%Trace
\newcommand{\tr}{ \mbox{Tr}  }

%Probability
% \newcommand{\prob}{\mathcal{P}}
% \newcommand{\prob}[1]{\mbox{Prob} \left[#1\right]}

%Identity operator
\newcommand{\id}{\mathbbm{1}}
% \newcommand{\id}{id}
\newcommand{\cj}{Choi-Jamio\l{}kowski }

%Set of unitaries
\newcommand{\UU}{\mathcal{U}}

%Entropies
\newcommand{\hmin}{ H_{\min} }
\newcommand{\hmax}{ H_{\max} }  
\newcommand{\renyi}{R\' enyi }

%Max eg value
\newcommand{\lmax}{ \lambda_{\max} }

%Operators
\newcommand{\cnot}{\mbox{ CNOT }}
\newcommand{\rank}{\mbox{rank }}
\DeclareMathOperator*{\supp}{supp}

%Matrices
% pmatrix, pmatrix* with usual brackets(.), vmatrix with determinant brackets|.|, Vmatrix with absolut value brackets||.||, bmatrix with [.], Bmatrix with {.}

%Matrix probabilities
\newcommand{\prob}[2]{
	\Pr_{#1}
	\left\{
		\begin{array}{l}
			#2
		\end{array}
	\right\}
	}

\newcommand*{\eps}{\varepsilon}
\newcommand*{\half}{\frac{1}{2}}
\newcommand*{\sqlf}{\frac{1}{\sqrt{2}}}


%Sets and spaces
\newcommand{\set}[1]{\left\{ #1 \right\}}
\newcommand{\bbN}{\mathbbm{N}}
\newcommand{\bbR}{\mathbbm{R}}
\newcommand{\N}{\mathbbm{N}}
\newcommand{\C}{\mathbbm{C}}

%TPCPMs
\newcommand*{\cA}{\mathcal{A}}
\newcommand*{\cB}{\mathcal{B}}
\newcommand*{\cC}{\mathcal{C}}
\newcommand*{\cE}{\mathcal{E}}
\newcommand*{\cF}{\mathcal{F}}
\newcommand*{\cG}{\mathcal{G}}
\newcommand*{\cH}{\mathcal{H}}
\newcommand*{\cI}{\mathcal{I}}
\newcommand*{\cL}{\mathcal{L}}
\newcommand*{\cM}{\mathcal{M}}
\newcommand*{\cS}{\mathcal{S}}
\newcommand*{\cT}{\mathcal{T}}
\newcommand*{\cX}{\mathcal{X}}
\newcommand*{\cY}{\mathcal{Y}}
\newcommand{\OO}[1]{\mathcal{O} \left( #1 \right) }

%Channel coding
\newcommand*{\enc}[1]{\operatorname{enc}_{#1}}
\newcommand*{\dec}[1]{\operatorname{dec}_{#1}}
\DeclareMathOperator*{\perr}{p_{err}}
\newcommand*{\bp}{\mathbf{p}}
\newcommand*{\rate}[1]{\operatorname{rate}_{#1}}
%\newcommand*{\enc}{\mathrm{enc}}
%\newcommand*{\dec}{\mathrm{dec}}
%\newcommand*{\perr}{p_{\mathrm{err}}}
\newcommand*{\rarrow}[1]{\raisebox{-1.1ex}{$\overrightarrow{\makebox[5em]{\vphantom{X}$#1$}}$}}

% Partial derivatives
\newcommand{\pd}[1]{ \frac{\partial}{\partial #1} }     %first derivative
\newcommand{\pdd}[1]{ \frac{\partial^2}{\partial {#1}^2 } }   %second derivative

%..................................

% Theorems 
\theoremstyle{plain}
\newtheorem{theorem}{Theorem}[section]
\newtheorem{lemma}[theorem]{Lemma}
\newtheorem{corollary}[theorem]{Corollary}
\newtheorem{claim}[theorem]{Claim}

\newtheorem{prop}{Proposition}

\theoremstyle{definition}
\newtheorem{definition}[theorem]{Definition}
\newtheorem{example}[theorem]{Example}

%..................................

% define boxed equations in all variants with package empheq with e.g. color blue:
\definecolor{myblue}{rgb}{.8, .8, 1}

\newlength\mytemplen
\newsavebox\mytempbox
\makeatletter
\newcommand\mybluebox{%
    \@ifnextchar[%]
       {\@mybluebox}%
       {\@mybluebox[0pt]}}

\def\@mybluebox[#1]{%
    \@ifnextchar[%]
       {\@@mybluebox[#1]}%
       {\@@mybluebox[#1][0pt]}}

\def\@@mybluebox[#1][#2]#3{
    \sbox\mytempbox{#3}%
    \mytemplen\ht\mytempbox
    \advance\mytemplen #1\relax
    \ht\mytempbox\mytemplen
    \mytemplen\dp\mytempbox
    \advance\mytemplen #2\relax
    \dp\mytempbox\mytemplen
    \colorbox{myblue}{\hspace{1em}\usebox{\mytempbox}\hspace{1em}}}
\makeatother


%..................................................

% Colours
\newcommand{\red}[1]{\textcolor{Red}{#1}}
\newcommand{\blue}[1]{\textcolor{Blue}{#1}}
\newcommand{\green}[1]{\textcolor{OliveGreen}{#1}}
\newcommand{\pink}[1]{\textcolor{Magenta}{#1}}
\newcommand{\yellow}[1]{\textcolor{Yellow}{#1}}
\newcommand{\orange}[1]{\textcolor{Orange}{#1}}

% Little things
\newcommand*{\ie}{i.e., }
\newcommand*{\etal}{et al.\ }
\newcommand*{\iid}{i.i.d.\ }

%side note
\newcommand{\flag}[1]{\green{ [#1]}}

%margin note
\newcommand{\Note}[1]{\marginpar{#1}}

%.................................................

% Referencing equations and figures
\newcommand{\eref}[1]{Eq.\ (\ref{#1})}
\newcommand{\fref}[1]{Fig.\ \ref{#1}}

%little figure for  "where on earth did that come from?" or "g, does this feel like reading someone else's code..."
% \newcommand{\alien}{\includegraphics[width=20pt]{fig/aliencode.jpg}\ }
%updated version of the logo
\newcommand{\alien}[1][3]{%
	\enspace
	\begin{tikzpicture}[scale=#1/20 pt]
		\useasboundingbox (-1,-1) rectangle (1,1);
		\draw (0,0) circle (1) [fill=blue!10];
		\filldraw	(-.5, .2) circle	(.1);
		\draw		(-.5, .3) circle	(.4);
		\filldraw	( .5, .2) circle	(.1);
		\draw		( .5, .3) circle	(.4);
		\draw		(-.1, .3) --		( .1, .3);
		\draw 		(-.3,-.55) .. controls (0.1,-.4) .. (.3,-.45);
	\end{tikzpicture}
  	\enspace
  	}

%little logo for approximation alert
% \newcommand{\alert}{\includegraphics[width=20pt]{fig/approximationalert.jpg} \ }
%updated version of the logo
\newcommand{\alert}[1][3]{%
	\enspace
	\begin{tikzpicture}[scale=#1/10 pt]
    	\filldraw[fill=green!10]	(0,0) ellipse (.9 and .5);
		\foreach \x/\y in {0/-.3,.7/-.3}
			\draw[cap=round,smooth,thick] (-.6 + \x,\y) -- (-.2 + \x,\y + .6) -- (-.1 + \x,\y) (-.465 + \x,\y + .2) -- (-.14 + \x,\y + .2);
% 		\node [red] at (0,0) {\sffamily \bfseries \itshape \scriptsize AA};
	\end{tikzpicture}
  	\enspace
  	}

%.................................................

% Tikz settings
\tikzset{
	% Feynman Diagrams
	photon/.style={decorate, decoration={snake}},
	% photondir/.style={decorate, decoration={snake}, segment amplitude=.8mm, segment length=4mm, line after s    nake=1.3mm, ->},
    photondir/.style={decorate, decoration={snake}, -stealth},
	fermion/.style={postaction={decorate},decoration={markings,mark=at position .55 with {\arrow{>}}}},
	gluon/.style={decorate,decoration={coil,amplitude=4pt,segment length=5pt}},
	boson/.style={dashed},
}

